\documentclass[8pt]{article}
\usepackage[legalpaper,margin=0.5in]{geometry}
\usepackage[utf8]{inputenc}
\usepackage[spanish]{babel}
\usepackage{multicol}
\usepackage{enumitem}
\usepackage{fancyhdr}
\setlength{\parindent}{0pt}
\setlength{\parskip}{1pt}

\pagestyle{fancy}
\fancyhf{}
\fancyhead[C]{\textbf{Examen de Programación Orientada a Objetos}}
\renewcommand{\headrulewidth}{0.4pt}

\begin{document}

\begin{multicols}{2}

\textbf{DATOS DEL ESTUDIANTE}\\
Nombre Completo: \rule{4cm}{0.5pt}\\
Carrera: \rule{4cm}{0.5pt}\\[2pt]

\textbf{DESCRIPCIÓN DEL EXAMEN}

\textbf{Clase Persona}

Se desea diseñar la clase Persona que tendrá las siguientes características:

\textbf{Atributos:}
\begin{itemize}[leftmargin=10pt, itemsep=0pt]
    \item Ci: entero
    \item CodigoProcedencia: cadena con dos caracteres [Pt, Or, Co, Ta, Ch, LP, SC, Be, Pa]
    \item Nombres: Cadena
    \item Apellidos: Cadena
\end{itemize}

\textbf{Comportamiento:}
\begin{itemize}[leftmargin=10pt, itemsep=0pt]
    \item ObtenerNombreCompleto(): Cadena
    \item VerDatos()
    \item ObtenerProcedencia(): Cadena
\end{itemize}

\textbf{Clase Estudiante (Herencia)}

\textbf{Atributos:}
\begin{itemize}[leftmargin=10pt, itemsep=0pt]
    \item Cu: cadena (Carnet universitario)
    \item Carrera: cadena
\end{itemize}

\textbf{Comportamiento:}
\begin{itemize}[leftmargin=10pt, itemsep=0pt]
    \item VerDatos() (sobrescrito)
    \item EsCompañeroDe(Estudiante): bool
\end{itemize}

\textbf{Clase Cuenta}

\textbf{Atributos:}
\begin{itemize}[leftmargin=10pt, itemsep=0pt]
    \item Nro Cuenta: Entero
    \item Saldo Actual: Real
    \item Propietario: Persona
\end{itemize}

\textbf{Comportamiento:}
\begin{itemize}[leftmargin=10pt, itemsep=0pt]
    \item Depositar(monto: real)
    \item Retirar(monto: real)
    \item Transferir(Cuenta)
    \item TransferirPromocion(Cuenta, PorcentajeIncremento)
\end{itemize}

\textbf{CRITERIOS DE EVALUACIÓN}

\textbf{Definición de la Clase Persona (30\%)}
\begin{itemize}[leftmargin=12pt, itemsep=0pt]
    \item Declaración de atributos y modificadores de acceso: 5pts
    \item Métodos de acceso y mutación: 5pts
    \item Dos constructores (por defecto y con parámetros): 5pts
    \item Método ObtenerNombreCompleto(): 5pts
    \item Método VerDatos(): 5pts
    \item Método ObtenerProcedencia(): 5pts
\end{itemize}

\textbf{Definición de la Clase Estudiante (20\%)}
\begin{itemize}[leftmargin=12pt, itemsep=0pt]
    \item Aplicación de herencia: 5pts
    \item Sobreescritura del constructor por defecto: 7pts
    \item Sobreescritura del método VerDatos: 8pts
\end{itemize}

\textbf{Definición de la Clase Cuenta (50\%)}
\begin{itemize}[leftmargin=12pt, itemsep=0pt]
    \item Atributos correctamente encapsulados: 10pts
    \item Métodos de depósito y retiro: 10pts
    \item Método Transferir(): 15pts
    \item Método TransferirPromocion(): 15pts
\end{itemize}

\textbf{NOTAS IMPORTANTES:}
\begin{itemize}[leftmargin=12pt, itemsep=0pt]
    \item Transferir() verifica si ambos estudiantes son compañeros de carrera
    \item TransferirPromocion() verifica si están en la misma región (Altiplano, Valles o Llanos)
    \item Imprimir mensajes informativos usando el nombre completo en las transferencias
    \item Códigos de procedencia: Pt=Potosí, Or=Oruro, Co=Cochabamba, Ta=Tarija, Ch=Chuquisaca, LP=La Paz, SC=Santa Cruz, Be=Beni, Pa=Pando
\end{itemize}

\columnbreak

\textbf{DATOS DEL ESTUDIANTE}\\
Nombre Completo: \rule{4cm}{0.5pt}\\
Carrera: \rule{4cm}{0.5pt}\\[2pt]

\textbf{DESCRIPCIÓN DEL EXAMEN}

\textbf{Clase Persona}

Se desea diseñar la clase Persona que tendrá las siguientes características:

\textbf{Atributos:}
\begin{itemize}[leftmargin=10pt, itemsep=0pt]
    \item Ci: entero
    \item CodigoProcedencia: cadena con dos caracteres [Pt, Or, Co, Ta, Ch, LP, SC, Be, Pa]
    \item Nombres: Cadena
    \item Apellidos: Cadena
\end{itemize}

\textbf{Comportamiento:}
\begin{itemize}[leftmargin=10pt, itemsep=0pt]
    \item ObtenerNombreCompleto(): Cadena
    \item VerDatos()
    \item ObtenerProcedencia(): Cadena
\end{itemize}

\textbf{Clase Estudiante (Herencia)}

\textbf{Atributos:}
\begin{itemize}[leftmargin=10pt, itemsep=0pt]
    \item Cu: cadena (Carnet universitario)
    \item Carrera: cadena
\end{itemize}

\textbf{Comportamiento:}
\begin{itemize}[leftmargin=10pt, itemsep=0pt]
    \item VerDatos() (sobrescrito)
    \item EsCompañeroDe(Estudiante): bool
\end{itemize}

\textbf{Clase Cuenta}

\textbf{Atributos:}
\begin{itemize}[leftmargin=10pt, itemsep=0pt]
    \item Nro Cuenta: Entero
    \item Saldo Actual: Real
    \item Propietario: Persona
\end{itemize}

\textbf{Comportamiento:}
\begin{itemize}[leftmargin=10pt, itemsep=0pt]
    \item Depositar(monto: real)
    \item Retirar(monto: real)
    \item Transferir(Cuenta)
    \item TransferirPromocion(Cuenta, PorcentajeIncremento)
\end{itemize}

\textbf{CRITERIOS DE EVALUACIÓN}

\textbf{Definición de la Clase Persona (30\%)}
\begin{itemize}[leftmargin=12pt, itemsep=0pt]
    \item Declaración de atributos y modificadores de acceso: 5pts
    \item Métodos de acceso y mutación: 5pts
    \item Dos constructores (por defecto y con parámetros): 5pts
    \item Método ObtenerNombreCompleto(): 5pts
    \item Método VerDatos(): 5pts
    \item Método ObtenerProcedencia(): 5pts
\end{itemize}

\textbf{Definición de la Clase Estudiante (20\%)}
\begin{itemize}[leftmargin=12pt, itemsep=0pt]
    \item Aplicación de herencia: 5pts
    \item Sobreescritura del constructor por defecto: 7pts
    \item Sobreescritura del método VerDatos: 8pts
\end{itemize}

\textbf{Definición de la Clase Cuenta (50\%)}
\begin{itemize}[leftmargin=12pt, itemsep=0pt]
    \item Atributos correctamente encapsulados: 10pts
    \item Métodos de depósito y retiro: 10pts
    \item Método Transferir(): 15pts
    \item Método TransferirPromocion(): 15pts
\end{itemize}

\textbf{NOTAS IMPORTANTES:}
\begin{itemize}[leftmargin=12pt, itemsep=0pt]
    \item Transferir() verifica si ambos estudiantes son compañeros de carrera
    \item TransferirPromocion() verifica si están en la misma región (Altiplano, Valles o Llanos)
    \item Imprimir mensajes informativos usando el nombre completo en las transferencias
    \item Códigos de procedencia: Pt=Potosí, Or=Oruro, Co=Cochabamba, Ta=Tarija, Ch=Chuquisaca, LP=La Paz, SC=Santa Cruz, Be=Beni, Pa=Pando
\end{itemize}

\end{multicols}

\end{document}
